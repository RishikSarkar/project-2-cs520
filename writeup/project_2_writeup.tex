\documentclass[11pt]{article}
\usepackage[tmargin=1in,lmargin=1in,rmargin=1in,bmargin=1.2in]{geometry}

\setlength{\parindent}{0pt}
\setlength{\parskip}{0.25em}
\usepackage{setspace}
\setstretch{1.1}

\usepackage{graphicx} % Required for inserting images

\title{CS520 Project 2 Data and Analysis}
\author{Aditya Girish, Rishik Sarkar}
\date{November 10 2023}

\begin{document}

\maketitle

\section{Bot Designs and Algorithms}

\subsection{Initial Setup}

This project is similar to the first one in that it involves a bot traversing through a grid maze of cells to save crew members while avoiding aliens. In our project version, we created a 30 x 30 NumPy grid containing 0s and 1s, with 0 representing open cells that the bot, aliens, and crew members can exist/move on and 1 representing closed cells (or walls). We have included the Python project within a single file, \emph{main.py}, which contains all the processes for probabilistic updates, sensing, bot and alien movement functions, version-specific bot simulations, and a thorough testing process for each bot that I shall outline in section 3. In our Python project, we use the following characters to represent each entity:

\begin{itemize}
    \item 0: Open Cell
    \item 1: Closed Cell
    \item 2: Bot
    \item 3: Alien(s)
    \item 4: Crew Member(s)
\end{itemize}

Furthermore, as defined in the project outline, there are 8 possible scenarios involving different combinations of bots, crew members, and aliens. Bots 1 and 2 involve 1 bot, 1 crew member, and 1 alien; Bots 3, 4, and 5 involve 1 bot, 2 crew members, and 1 alien; and Bots 6, 7, and 8 involve 1 bot, 2 crew members, and 2 aliens. Each scenario involves certain similarities and differences in the grid traversal algorithm for the bot, but all of them involve deterministic and probabilistic processes that I shall outline below.

\subsection{One Alien, One Crew}

As mentioned earlier, Bot 1 and Bot 2 both involve 1 bot, 1 alien, and 1 crew member within a 30 x 30 grid maze.

\subsubsection{Bot 1}

Bot 1 utilizes a crew sensor and an alien sensor at every single time step to figure out if there is a crew member or alien nearby. On top of that, the probabilities of the crew member and alien being at a particular cell are updated every time the bot and alien move. Bot 1 stores the probabilities of the crew and alien being in a certain cell (i, j) within two probability matrices (represented as a dictionary of open cells + bot's current location) \textit{crew\_matrix} and \textit{alien\_matrix}.

\subsubsection{Bot 2}

\subsection{One Alien, Two Crew}

\subsubsection{Bot 3}

\subsubsection{Bot 4}

\subsubsection{Bot 5}

\subsection{Two Aliens, Two Crew}

\subsubsection{Bot 6}

\subsubsection{Bot 7}

\subsubsection{Bot 8}



\section{Probability Models and Updates}

\subsection{One Alien, One Crew}

\subsubsection{Bot 1}

\subsubsection{Bot 2}

\subsection{One Alien, Two Crew}

\subsubsection{Bot 3}

\subsubsection{Bot 4}

\subsubsection{Bot 5}

\subsection{Two Aliens, Two Crew}

\subsubsection{Bot 6}

\subsubsection{Bot 7}

\subsubsection{Bot 8}



\section{Performance Evaluation}

\subsection{Average Moves to Save All Crew}

\begin{itemize}
    \item Bot 1 vs. Bot 2
    \item Bot 3 vs. Bot 4 vs. Bot 5
    \item Bot 6 vs. Bot 7 vs. Bot 8
\end{itemize}

\subsection{Probability of Avoiding Alien and Saving Crew}

\begin{itemize}
    \item Bot 1 vs. Bot 2
    \item Bot 3 vs. Bot 4 vs. Bot 5
    \item Bot 6 vs. Bot 7 vs. Bot 8
\end{itemize}

\subsection{Average Number of Crew Saved}

\begin{itemize}
    \item Bot 1 vs. Bot 2
    \item Bot 3 vs. Bot 4 vs. Bot 5
    \item Bot 6 vs. Bot 7 vs. Bot 8
\end{itemize}

% \begin{figure}
%     \centering
%     \includegraphics[width=0.75\linewidth]{placeholder}
%     \caption{D=15, K=7 (Win Rate)}
%     \label{fig:sim-2-win}
% \end{figure}

\section{The Ideal Bot}



\section{Bonus}



\section{Final Thoughts and Relevant Information}



\end{document}